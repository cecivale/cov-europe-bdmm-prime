\documentclass{article}
\usepackage[a4paper, total={6in, 8in}]{geometry}

\usepackage[T1]{fontenc}
\usepackage[utf8]{inputenc}
\usepackage{palatino}

\usepackage{fancyhdr}
 \fancypagestyle{plain}

\usepackage{graphicx}
\usepackage{enumerate}

% Header
\fancyhf{}
\rhead{\begin{picture}(0,0) \put(-56.7,0){\includegraphics[width=20mm]{dbsse}} \end{picture}}
\lhead{\begin{picture}(0,0) \put(0,0){\includegraphics[width=25mm]{ethz}} \end{picture}}
%\chead{Master's Thesis Project Outline}
\renewcommand{\headrulewidth}{0pt}
 
 % Foot
\rfoot{Basel, \today}


% Title

\usepackage{etoolbox}
\makeatletter
\patchcmd{\@maketitle}{\begin{center}}{\begin{flushleft}}{}{}
\patchcmd{\@maketitle}{\begin{tabular}[t]{c}}{\begin{tabular}[t]{@{}l}}{}{}
\patchcmd{\@maketitle}{\end{center}}{\end{flushleft}}{}{}
\makeatother

\title{A comprehensive study of the phylodynamics of SARS-CoV-2 in Europe }
\author{Cecilia Valenzuela Agüí \\  \\  Supervised by Prof. Dr. Tanja Stadler, Dr. Timothy Vaughan \\ Computational Biology Group }
\date{CBB Master's Thesis Project Outline \\ September 1, 2020 - March 16, 2021 }


% Document
\begin{document}

\maketitle
\begin{flushleft}
\rule{\textwidth}{0.5pt}

\vspace{10mm}

This project will focus on the spatial dynamics of the early spread of SARS-CoV-2 in Europe. We will apply a novel approach based on the Multi-type Birth Death phylodynamic model to infer structured population dynamics jointly with between-subpopulation transmission rates from viral genome sequences. The inferred epidemic trajectories for the combined outbreak responsible for the observed sequence data will allow us to better understand the entry into and early spread of SARS-CoV-2 in Europe.

\vspace{4mm}

For the analysis, we will use the software package BEAST2, a tool for Bayesian evolutionary analysis of molecular sequences using Markov chain Monte Carlo (MCMC). The sequence data will be gathered from the publicly accesible GISAID database. 

\vspace{4mm}

\textbf{Project tasks}

\begin{enumerate}
\item Create a current and representative sequence alignment for the early phase of COVID-19 pandemic from available SARS-CoV sequences . This includes sequence selection, curation and alignment.
\item Use travel and geographical data as the basis for informative priors on migration rates, this could significantly improve the precision and accuracy of the results.
\item Reconstruction of a partial transmission tree of the early pandemic, including inferred geographic location of ancestral lineages.
\item Inference of multi-type epidemic trajectories, allowing us to directly estimate the number SARS-CoV-2 of introductions into a country.
\end{enumerate}

\end{flushleft}
\end{document}




