\chapter{Introduction}

We have many sources of information, but all of them are imperfect in some aspect. Bayesian phylodynamics incorporates different beliefs and data to get a detailed picture of the epidemic. Genetic sequences as an objective source of information about virus evolution and transmision together with viral phylogenetics, transpotation data and geographic data to understand the dynamics and case counts to guide the magnitutde of the numbers.

We lack absolute numbers of the pandemic time resoluted without reporting issues dependent. We lack understading of the dynamics, from where to where, how often, how long.

We can use travel data as a proxy, or we can incorporate it in a model. Lemey et al. Difference we get absolute numbers, not only the rates. And we use BD model instead of coalescent, better to describe beginning of a epidemic. While due to the complexite and time consuming model we dow scaled to europe and only thhe initial phase with the first countries to experiment relevant sars cov 2 epidemics. 

the first reported cases were in China in early December \cite{},


In a phylodynamic analysis, we aim to integrate genetic and epidemiological information to understand pathogen evolution and transmission dynamics. We can learn about the epidemic spread and the interactions between hosts from the imprint that these events leave in virus phylogenies. This is possible for RNA viruses, and in specific for SARS-CoV-2, because virus genetic evolution and epidemic processes are in the same time scale \cite{Grenfell2013} \cite{Volz2014}.  More precisely, the field of phylodynamics studies how phylogenetic trees are being generated and infers the population parameters behind that process. 


\todo[inline]{TODO introduction}

\begin{itemize}
\item Importance of understanding the spread of the virus to prevent future outbreaks 

\item About phylodynamics/phylogeographics? Use of genetic sequences as a source of information combined with other sources of information as travel data

\item About what we know of the introduction of sars-cov-2 in Europe and early dynamics till 8 March

\item About what we know of the case counts in Europe? Can sequences help?

\item Introduce/formulate the questions: case counts, first introductions, ?migration vs within region transmission?, migration patterns, ?border closures.
\end{itemize}