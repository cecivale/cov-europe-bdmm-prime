\chapter{Introduction}

We have many sources of information, but all of them are imperfect in some aspect. Bayesian phylodynamics incorporates different beliefs and data to get a detailed picture of the epidemic. Genetic sequences as an objective source of information about virus evolution and transmision together with viral phylogenetics, transpotation data and geographic data to understand the dynamics and case counts to guide the magnitutde of the numbers.

We lack absolute numbers of the pandemic time resoluted without reporting issues dependent. We lack understading of the dynamics, from where to where, how often, how long.

We can use travel data as a proxy, or we can incorporate it in a model. Lemey et al. Difference we get absolute numbers, not only the rates. And we use BD model instead of coalescent, better to describe beginning of a epidemic. While due to the complexite and time consuming model we dow scaled to europe and only thhe initial phase with the first countries to experiment relevant sars cov 2 epidemics. 

