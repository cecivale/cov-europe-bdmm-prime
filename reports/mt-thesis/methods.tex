\chapter{Material and Methods}

\section{SARS-CoV 2 genome data set}

We create a dataset of 360 genetic sequences from December 2019 to March 8 2020, obtained from publicly available data on GISAID \cite{GISAID} (accessed on November 2020). We follow the Nextstrain workflow for the curation of the sequences \cite{nextstrain}. Sequences with incomplete collection date, less than 27.000 bases in length or with more than 3.000 unknown bases are omitted. Also, sequences from known clusters of transmission or from the same patient are excluded from the analysis. The resulting dataset of x sequences are aligned with MAFFT. The beggining and the end of the alignment are masked respectively by 100 and 50 sequences as well as sites  13402, 24389 and 24390, identified by Nextstrain as prone to sequencing errors.

To focus on the early dynamics in Europe we select sequences from China, origin of the epidemic, France and Germany, the European countries with the earliest cases, and Italy and Spain, the European countries with the biggest outbreaks in March. To take into account the dynamics in other areas of Europe we include a group of 50 sequences from other European countries. We limit our sample of Chinese genomes to sequences until January 23, the starting date of the lockdown in Hubei. Due to the large (and unprecendented) number of available genetic sequences for SARS-CoV-2, we need to subsample the alignment. Each sequence is subsampled with a probability of having a reported case in that country the day of sample collection, inversely weigted by the probability of having a sequence in GISAID that day from the country. With this subsampling protocol, we aim to get a constant sampling proportion accross the full period for each country. 

\section{The multitype birth death model}


In a phylodynamic analysis, we aim to integrate genetic and epidemiological information to understand pathogen evolution and transmission dynamics. We can learn about the epidemic spread and the interactions between hosts from the imprint that these events leave in virus phylogenies. This is possible for RNA viruses, and in specific for SARS-CoV-2, because virus genetic evolution and epidemic processes are in the same time scale \cite{Grenfell2013} \cite{Volz2014}.  More precisely, the field of phylodynamics studies how phylogenetic trees are being generated and infers the population parameters behind that process. 
\todo{Introduction?}


To study the early dynamics of SARS-CoV-2 in the European countries, we use a simplified version of the multitype birth-death model described in \cite{Kuhnert2016}, following the analysis in \cite{Nadeau2020}. Birth death models are compartmental population models with high flexibility that describe the process of epidemiological transmission. The stochastic formulation of these models are used in phylodynamic analyses \cite{Stadler2012}. In the multitype version, we consider a structured population in types or subpopulations with characteristic within-subpopulation dynamics and migrations between them.

The process starts with one infected host in one of the subpopulations, e.g in subpopulation $i$, who can infect another individual at rate $\lambda_i$, become uninfectious at rate $\mu_i$ by death or recovery, migrate to another deme $j$ at rate $m_{ij}$ or be sequenced at rate $\psi_i$ to become part of the phylogenectic tree. This process depicts the full transmission dynamics and specifically, the generation of the transmission tree that we observed from our sequence data. 

We parameterize our model in terms of the effective reproductive number $R_i = \frac{\lambda_i}{\mu_i + \psi_i}$, a key value in epidemic control and understanding, the rate of becoming uninfectious $\delta_i = \mu_i + \psi_i$ and the probability of and individual to be sequenced $s_i = \frac{\psi_i}{\mu_i + \psi_i}$.


\section{Multi-typed trees and epidemic trajectories}


\section{Inference of epidemic dynamics and trajectories}

Under this model, we are able to compute the likelihood of the multitype birth-death parameters for a given phylogenetic tree. This likelihood is derived in \cite{Kuhnert} by considering the probability of an individual evolved as observed in the tree.This derivation is analogous to the work in Stadler and Bonhoeffer (2013), which is based on ideas from (Maddison et al. 2007)

This phylodynamic likelihood is the tree prior in our Bayesian inference. We want to estimate the overall posterior distribution of the phylogenetic tree with ancestral locations $\mathcal{T}_c$, the epidemic trajectory $\mathcal{E}$, the substitution model parameters $\theta$ and the multi-type birth death parameters $\eta$:

\begin{equation}
P(\mathcal{T}_c, \mathcal{E}, \theta, \eta | A, S) = \frac{\overbrace{P(A | \tau, \theta)}^{\text{tree likelihood}} \overbrace{P(\tau, S | \eta)}^{\text{tree prior}} \overbrace{P(\theta, \eta)}^{\text{parameters prior}}}{\underbrace{P(A, S)}_{\text{marginal likelihood}}}
\end{equation}

where A is the sequence alignment and S the observed sequence times.


From the phylogenetic tree and the phylodynamics parameters we use Stochastic mapping to infer the ancestral traits and then to infer the epidemic trajectories.

$P(Tc | T, P)$

New method to infer epidemic trajectories!!

$P(E | Tc, P)$

Stochastic mapping of ancestral traits

Read paper and tim presentation

Stochastic mapping of epidemic trajectories

From the phylogenie with ancestral locations and the set of phylodynamic parameters we can simulate epidemic trajectories using an algorithm similar to Gillespie (with tau leaping and particle filtering? the subsampling). And most improtant, we can compute the probability of that specific trajectory given those parrameters and tree. 

$P(E|Tc, P)$

Any trajectories without the events that are respresent in the tree has probability 0, so to avoid the simualtion of this trajectories, no time effcieint, we will enforce the events that arer represented in the tree (and will include a weigth). Also to avoid low probabilities trajectories, we will importance subsample one trajectory after a certain time of simulation according to the weigths detemrined by its probabilities. 


\section{Modeling the early dynamics of SARS-CoV-2 in Europe}

In our case, we have 6 demes, one for each of the countries/geographic location. A birth represents a transmission event of covid-19, a death event is an individual becoming uninfectious by overcoming the disease, being isolated or dying. From the birth rate and death rate we can estuimate the R0, key vallue in the epidemic understanding and control. A migration event happens when an individual travels from one country to another, we know that is an important way of covid-19 spreading and we are interested on the dynamics. 

We focus our analysis in the early spread in Europe of SARS-CoV-2, therefore we constrain the analysis to the period from the origin of the epidemic to March 8, when the first European lockdown in the Lombardy region started. During this time, for the European countries we expect an unimpeded spread of the virus that could be described by a constant R0 particular to each deme. However, the migration rates could have changed during these months due to the increase public awareness, so we will consider a different migration rate for origin to 23 jan, 23 jan to end of february and 1 to 8 of March. We assume a constant sampling proportion different for each of the demes, to account for difference in the sequencing efforts. The becoming uninfectious rate we will fixed in 36.5, assuming that one individual is infectious for a exponential time distirbuted with mean 10.


\section{Incorporating travel data}

To incorporatet travel data in the model we have use a GLM model described in Lemey et al 2009. We use a generalized linear model to describe the migration matrix KxK, where the rpedictors are the number of dayly flight from one country to another from EUROSTATS, the average distance between the centrroids of the country and the population sizes from origin and destinations. The predictors are log transformed, we add a pseudocount to make them all positive, and standardized as described in ..

$mij = c exp(\sum_{i = 1}^p \delta_i \beta_i xij)$

The coefficients beta describe the effect size of the predictors in the migrations rate and the delta coefficients act as a model selection variables, taking 0-1 values including orr not that specific predictor in the model.


\section{GLM predictos data}

EUROSTAT, countries considers. Distance and population datasets.


\section{Bayesian inference of phylodynamic parameters}

We use bayesian inference and Metropolis Hasting Monte Carlo algorith to infer our phylodynamics parameters and phylogeneit tree. We use the following priors

Clock rate and substitution model. Fixed to 8x10-4, HKY 4 gamma categories with priors...
BDMM-Prime epi parameterization:
R0
Become uninfectious rate
Sampling rates
Migration rates


\subsubsection{R0}
R0 constant except for China (and Italy? maybe not in the main analysis).

Prior Log Normal (0.8, 0.5) Median 2.2, 95IQR 0.8 to 5.9.
(Lai A. 2020 prior log normal (0,1) median 1.0 0.1-7.10)

- Main analysis: We fix China R0 to 2.7 (23 jan) 1.3 (10 feb) 0.8
- Use a decreasing paramater as Sarah with migration rate? But then we would need to add sequences.
- Or instead of fixing it we can put a strong informative prior.
- Allow Italy R0 to change March 1.


Meta-analyses R0:

R0 in China (3.14,95\%CI,2.40–4.09).\cite{Billah2020}

R0value 2.90 (2.32, 3.63
 2.9 (95\% CI: 2.1–4.5)\cite{Park2020}


Periods:
until 23 jan R0 2.9 
23 jan - 10 feb 1.1 dashboard cevo
10 feb - 8 mar 0.43 dashboard cevo



Become uninfectious rate same for all demes, fixed to 36.5.
Migrations among demes, 3 epochs.
Sampling proportion 0 before first sample. Upperbounded by number of cases.

Implementation BEAST 2.6.3 and BDMM-Prime package.
We run 5 chains of 10e7 iterations witth different seeds.

\section{Processing of epidemic trajectories}
In our case, for each set of parameters and tree we will simulate 10000 trajectories with an epsilon of 0.3 and select one of them with importance sampling.
We have one trtajectory for each step in the MCMC. We subsampled x. 

\section{Data availability}
GISAID data table.

\section{Code availability and analyses reproducibility}
All the code is availables at ...
We have implement the full workflow with Snakemake so its fully reproducible.

